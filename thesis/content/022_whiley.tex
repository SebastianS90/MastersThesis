% !TeX root = ../main.tex
\section{Whiley}\label{section:background-whiley}

This section provides an introduction to the \whiley programming language.
We cover the language's main goals and all parts that are relevant for our extension to \whiley.

A more complete specification of \whiley is available in the official \emph{Whiley Language Specification} \cite{WLS}.
It introduces \whiley as follows:

\begin{quote}
``Whiley is a hybrid imperative and functional programming language designed to produce programs with as few errors as possible.
Whiley allows explicit specifications to be given for functions, methods and data structures, and employs a verifying compiler to check whether programs meet their specifications.'' \cite[p.~7]{WLS}
\end{quote}


\subsection{Design}

\begin{figure}[h]
\centering
% !TeX root = ../../main.tex
\begin{tikzpicture}[>=stealth',shorten >=1pt]
\tikzstyle{filebox} = [anchor=south west,rectangle,draw,fill=white,minimum width=25mm,minimum height=2cm]
\tikzstyle{whileybox} = [filebox,minimum width=23mm,minimum height=2cm]
\tikzstyle{wyilbox}   = [filebox,minimum width=19mm,minimum height=2cm]
\tikzstyle{jvmbox}    = [filebox,minimum width=18mm,minimum height=11mm]
\tikzstyle{cbox}      = [filebox,minimum width=18mm,minimum height=1cm]

\node (whiley3) at (2mm,2mm) [whileybox] {};
\node (whiley2) at (1mm,1mm) [whileybox] {};
\node (whiley1) at (0mm,0mm) [whileybox] {};
\node at (whiley1.north west) [anchor=north west,align=left] {%
	\scriptsize\whileyinline{method main:}\\%
	\scriptsize\vspace*{-1mm}\ldots\\%
	\scriptsize\vspace*{-1mm}\ldots\\%
	\scriptsize\vspace*{-1mm}\ldots%
};
\node[anchor=north] at (whiley1.south) {\footnotesize\texttt{prog.whiley}};

\node (parser) at ($(whiley1.east)+(8mm,0)$) [anchor=west,ellipse,draw,align=center] {%
	\footnotesize Parsing\\%
	\footnotesize Type Checker%
};

\node (wyil3) at ($(parser.east |- whiley1.south west)+(8mm,2mm)$) [wyilbox] {};
\node (wyil2) at ($(wyil3.south west)-(1mm,1mm)$) [wyilbox] {};
\node (wyil1) at ($(wyil2.south west)-(1mm,1mm)$) [wyilbox] {};
\node at (wyil1) [align=center] {%
	\footnotesize WyIL\\%
	\footnotesize bytecode%
};
\node (wyilcaption) at (wyil1.south) [anchor=north] {\footnotesize\texttt{prog.wyil}};

\node (verifier) at ($(wyilcaption.south)+(0,-6mm)$) [anchor=north,ellipse,draw] {\footnotesize Verifier};

\node (wyjc) at ($(wyil1.east)+(8mm,8mm)$)  [anchor=west,ellipse,draw,minimum width=2cm] {\footnotesize WyJC};

\node (wycc) at ($(wyil1.east)+(8mm,-12mm)$) [anchor=west,ellipse,draw,minimum width=2cm] {\footnotesize WyCC};

\node (jvm3) at ($(wyjc.east)+(8mm,2mm)$) [jvmbox,anchor=west] {};
\node (jvm2) at ($(jvm3.south west)-(1mm,1mm)$) [jvmbox] {};
\node (jvm1) at ($(jvm2.south west)-(1mm,1mm)$) [jvmbox] {};
\node at (jvm1) [align=center] {%
	\footnotesize JVM\\%
	\footnotesize bytecode%
};
\node (jvmcaption) at (jvm1.south) [anchor=north] {\footnotesize\texttt{prog.class}};

\node (c3) at ($(wycc.east)+(8mm,2mm)$) [cbox,anchor=west] {};
\node (c2) at ($(c3.south west)-(1mm,1mm)$) [cbox] {};
\node (c1) at ($(c2.south west)-(1mm,1mm)$) [cbox] {};
\node at (c1) {\footnotesize C code};
\node (ccaption) at (c1.south) [anchor=north] {\footnotesize\texttt{prog.c}};

\draw[->] (whiley3.east |- whiley1.east) -- (parser.west); 
\draw[->] (parser.east) -- (wyil1.west);
\draw[->] ($(wyil3.east |- wyil1.east)+(0,1mm)$) -- (wyjc.west);
\draw[->] ($(wyil3.east |- wyil1.east)-(0,1mm)$) -- (wycc.west);
\draw[->] (wyjc.east) -- (jvm1.west);
\draw[->] (wycc.east) -- (c1.west);
\draw[->] (wyilcaption.south) -- (verifier.north);

\draw [decorate,decoration={brace,amplitude=10pt,mirror},yshift=-3mm]
($(wycc.west |- ccaption.south)-(1mm,0)$) -- ($(c3.east |- ccaption.south)+(1mm,0)$) node [midway,below=4mm] 
{\footnotesize backends};

\end{tikzpicture}

\caption{Design of the \whiley compiler}
\label{figure:whiley-compiler-design}
\end{figure}

\whiley consists of different components as shown in \myref{Figure}{figure:whiley-compiler-design}.
Programs are written in the \whiley language and stored as \texttt{.whiley} files.
Compilation happens in two steps:
The program is first translated to the \emph{Whiley Intermediate Language (WyIL)}.
The WyIL program is then passed to a \emph{backend} which generates code for a destination language.
Currently, the only usable backend is the \emph{Whiley to Java Compiler (WyJC)}, which generates byte code for the \emph{Java Virtual Machine (JVM)}.
There is a backend to generate \emph{C} code, but it does not free any memory.
Therefore it is not yet usable for large or long-running programs, or for small embedded systems that do not have a large amount of memory.

Furthermore, the program can be \emph{verified}.
This is an optional but recommended step during compilation.
Verification is presented in \myref{section}{section:whiley-verification}.


\subsection{Functions, Methods and Specification}
\whiley distinguishes between \emph{functions} and \emph{methods}.
The former one are \emph{pure}: a function will always return the same result if the same arguments are given.
Furthermore, functions cannot have any side-effects.
Methods do not obey these restrictions.
The can dereference pointers and issue I/O operations.
It is not allowed to call a method from a function.

\whiley uses a call-by-value semantics.
The function parameters can therefore be assumed to be a deep copy of the arguments given by the caller.
Furthermore, all values are copied on assignment to other variables.
Even when a function called with an array \whileyinline{a} assigns \whileyinline{a[0] = 2}, it does not affect the caller's array.
This is similar to the types in \rust that are marked with \rustinline{Copy}.
There is no aliasing for values in \whiley, unless references are used.
References will be introduced in \myref{section}{subsection:background-whiley-references}.

Specification of functions and methods in \whiley follows a simple precondition-postcondition schema.
Both functions and methods can be specified.
If a precondition is given, then it is not allowed to call that function or method unless the precondition is satisfied for the given arguments.
After executing the function or method, the postcondition must be satisfied.
Both is statically checked by the compiler, provided that it is used with the \texttt{-verify} flag.

Consider a function that takes an array of numbers and yields the position of the maximum entry.
If the maximum appears more than once in our array, then the first position shall be given.
But what should happen if the array is empty?
In \whiley, you typically specify that the function must not be called with an empty array.
A possible specification is as follows:

\begin{whileycodec}{background-whiley-max-spec}{Specification for our max function}
function max(int[] input) -> (int result)
requires |input| > 0
ensures result >= 0 && result < |input|
ensures all { j in 0..|input| | input[j] <= input[result] }
ensures all { j in 0..result  | input[j] <  input[result] }:
	// ...
\end{whileycodec}

Line~1 declares the parameters and the return type.
Line~2 states the precondition: length of the given array must be greater than \whileyinline{0}.
In line~3 to 5 we give several postconditions.
All \whileyinline{ensures} clauses are semantically connected as conjunctions.
We can reference the returned value by the name declared together with the return type, \whileyinline{result} in our case.
The first postcondition specifies that the returned result is a valid index in the given array.
The second condition ensures that the returned position actually holds a maximal value, i.e. all array values are less or equal than the maximal value.
The last condition ensures that we return the first maximal position: all smaller positions hold values that are smaller (but not equal).


\subsection{Verification}\label{section:whiley-verification}

A simple implementation for the program specified in \lstref{background-whiley-max-spec} is as follows:
\begin{whileycode}
int result = 0
int i = 0
while (i < |input|):
	if input[i] > input[result]:
		result = i
	i = i + 1
return result
\end{whileycode}

\FloatBarrier

The idea is as follows: we start with \whileyinline{0} as position holding a maximal value.
Then we iterate through the array.
If we find a position with a greater value, then we remember that new position as holding the maximal value.

But if we enable verification, then the program will not compile.
\whiley's verifier can handle sequential program flow quite well.
It applies \emph{Hoare Logic} and \emph{strongest postcondition transformation}.
This does not work for loops.
We have to give a \emph{loop invariant} to help the theorem prover that internally does the verification.
For values that are not assigned inside the loop we do not need to give any additional invariants.
But our loop contains assignments to \whileyinline{i} and \whileyinline{result}, so we need to come up with an invariant that somehow contains these values.

\begin{figure}[!b]
\begin{whileycodec}{background-whiley-max-complete}{Specified and verified max function}
function max(int[] input) -> (int result)
requires |input| > 0
ensures result >= 0 && result < |input|
ensures all { j in 0..|input| | input[j] <= input[result] }
ensures all { j in 0..result  | input[j] <  input[result] }:
    result = 0
    int i = 0
    while (i < |input|)
    where i >= 0 && i <= |input|
    where result >= 0 && result < |input| && result <= i
    where all { j in 0..i       | input[j] <= input[result] }
    where all { j in 0..result  | input[j] <  input[result] }:
        if input[i] > input[result]:
            result = i
        i = i + 1
    return result
\end{whileycodec}
\end{figure}

\Lstref{background-whiley-max-complete} presents a working implementation that passes verification.
Loop invariants are given using the \whileyinline{where} keyword.
Multiple invariants are semantically connected as conjunctions.
Loop invariants must be satisfied before entering the loop and after executing the loop body.

Our first invariant states that the loop variable \whileyinline{i} is in bounds of the \whileyinline{input} array, or equal to the arrays length.
The latter one is for the case after the last loop iteration.
One job of the verifier is to ensure that each array access is in bounds.
This invariant combined with the loop condition ensures that \whileyinline{input[i]} is always in bounds.

The second invariant states that \whileyinline{result} is always a valid index and it is smaller or equal to the current loop variable \whileyinline{i}.
On entry, both variables are \whileyinline{0}, which satisfies that condition because the array is not empty.
If \whileyinline{result} is updated, then it is set to the current index which is known to be in bounds.
The invariant is therefore restored after each loop iteration.

The last two invariants are very similar to the function's postcondition.
The only difference is that we consider only values up to position \whileyinline{i}.
Therefore these conditions trivially hold on loop entry.
It is easy to see that the loop restores both conditions with every iteration.

After exiting from the loop, it is known that \whileyinline{!(i < |input|)} holds.
Furthermore, \whileyinline{i <= |input|} from our invariants still holds.
The combination of both yields \whileyinline{i == |input|}.
Combined with invariant from line~11 we can satisfy the postcondition in line~4.
The remaining postconditions are directly contained in our loop invariants.

\FloatBarrier

\subsubsection*{Assert and Assume}
Besides loop invariants, \whiley offers two statements that interact with verification: \whileyinline{assert} and \whileyinline{assume}.
Both statements will check at runtime that the given condition is satisfied.
At verification time, i.e. on compile time if the \texttt{-verify} flag is given, they influence the verifier.
If \whileyinline{assert} is used, then the verifier must find a proof for that condition.
The condition will then automatically hold for all program executions.

On the other hand, \whileyinline{assume} inserts the given condition as an additional assumption at that position.
It can be used to omit verifying at some places, e.g. when the theorem prover cannot find a suitable proof.
A wrong use of \whileyinline{assume} can compile unsound programs.
The program still aborts at runtime because of the invalid assumption, but it passes the verifier at compile time although it might contain an out-of-bounds array access:

\begin{whileycode}
int[] a = [1, 1, 2, 3, 5, 8]
assert |a| == 6

int i = 7
assume i >= 0 && i < |a|
int x = a[i] // out of bounds check effectively disabled
\end{whileycode}


\subsection{Structural Typing and Records}

Another main feature of \whiley is \emph{structural typing}.
To understand this class of type systems, we look at the differences between nominal and structural typing.
In nominal typing, an entity's type is defined by the use of a specific name.
Even if two type declarations have exactly the same definition, they declare two different types.
In structural type systems, an explicit type declaration is not needed.
The type of an entity is derived from its structure.
Two types with equivalent structures are considered to be the same.

One example for a use of structural types in \whiley are \emph{records}.
A record is a compound that contains \emph{fields}.
Each field has a name and a type.
There is no specific order among record fields.
Consider the following program:

% strange bug that code must not start with opening brace unless escaped...
\begin{whileycode}\
{int x, int y} rec1 = {y: 7, x: 8}
{int y, int x} rec2 = {x: 8, y: 7}
assert rec1 == rec2

assert rec1.x == 8
rec1.x = 9
assert rec1 != rec2
\end{whileycode}

Focus on the first half.
We assign two local variables with records.
Although the order of fields does not match, these assignments are valid because records are unordered.
Furthermore, the assertion in line~3 is correct, i.e. both records have the same type and value.
The second half of the program illustrate how we can read and write to record fields.

Record types can be \emph{open}.
An open record allows to have additional fields that are not specified by the type itself.
Consider the following code snippet:

\begin{whileycode}\
{int x, int y, ...} rec1 = {y: 7, x: 8, z: 9, hello: "world"}
{int y, ...} rec2 = rec1
\end{whileycode}

We declare a local variable \whileyinline{rec1} as open record and assign a value.
The assigned record actually has more fields than the type, but it is allowed because the type is an open record.
In line~2, we assign our record to another local variable \whileyinline{rec2}.
Interestingly, the type of \whileyinline{rec2} is not at all the same as \whileyinline{rec1}.
\whiley uses the type's structure for \emph{subtyping} and considers \whileyinline{{int x, int y, ...}} to be a subtype of \whileyinline{{int y, ...}}.
This is referred to as \emph{width subtyping} in the literature.

To understand subtyping in \whiley, consider for each type the set of all values it describes.
A type \texttt{T1} is subtype of \texttt{T2} if the set of all values with type \texttt{T1} is a subset of the values of \texttt{T2}.
The type of \whileyinline{rec1} in the program above describes the set of all records that contain at least the fields \whileyinline{int x} and \whileyinline{int y}.
The type of \whileyinline{rec2} describes the set of all records that contain at least the field \whileyinline{int y}.
It imposes less restrictions to its values and therefore is a superset of the set described by \whileyinline{rec1}'s type.

Two types are the same if they are subtypes of each other.
This is equivalent to stating that their sets of values are the same.


\subsection{Union Types}

\whiley's type system allows us to define types as unions of other types.
A union contains all values that are present in any of its components.
One possible usage is to use \whiley's \whileyinline{null} type in a union.
The type \whileyinline{null} has exactly one value, namely \whileyinline{null}.
It is meant to express the absence of something.
Used in a union, it can make the type somehow optional.
Consider a function that takes an array and an element.
It should find a position in that array where the given element occurs.
The function can be specified and implemented as follows:

\begin{whileycode}
function indexOf(int[] a, int x)->(int|null r)
ensures r is null ==> no { j in 0..|a| | a[j] == x }
ensures r is int  ==> a[r] == x:
	int i = 0
	while (i < |a|)
	where i >= 0 && i <= |a|
	where no { j in 0..i | a[j] == x }:
		if (a[i] == x):
			return i
		i = i + 1
	return null
\end{whileycode}

The return type is a union of \whileyinline{int} and \whileyinline{null}.
The specification ensures that \whileyinline{null} is only returned when the given element is not contained in the array.


\subsection{Type Declarations and Recursive Types}

Structural types can be quite long and writing the type every time it is needed might clutter your program.
\emph{Type declarations} allow to define a name which can be used as alias for its declared type.
Both the name and its structural type refer to the same type:
\begin{whileycode}
type Point is {int x, int y}

method main():
	Point rec1 = {x: 8, y: 7}
	{int y, int x} rec2 = rec1
\end{whileycode}

Type declarations can carry \emph{type invariants}.
They constrain the declared type to only contain values that satisfy the given boolean expressions.
For example, the following snippet specifies natural numbers and even numbers:
\begin{whileycode}
type nat  is (int x) where x >= 0
type even is (int x) where x % 2 == 0
\end{whileycode}

Type declarations allow to specify \emph{recursive types}.
A recursive type directly or indirectly refers to itself within its definition.
The following program defines and uses a recursive type \whileyinline{LinkedList}.

\begin{whileycode}
type LinkedList is null | {int head, LinkedList tail}

method main():
	LinkedList list =
	    {head: 1, tail: {head: 2, tail: {head: 3, tail: null}}}
\end{whileycode}

Although recursive types are not uncommon in other programming languages, they impose some challenges when it comes to structural typing.
Consider the following type definitions:

\begin{whileycode}
type A is null | B
type B is {tail: C, head: int}
type C is {int head, A tail} | null
\end{whileycode}

After having a close look on these types we can see that \whileyinline{A} and \whileyinline{C} are structurally the same as \whileyinline{LinkedList}.
\whiley therefore considers \whileyinline{A}, \whileyinline{C} and \whileyinline{LinkedList} to be the same type.
Furthermore, all three are a subtype of type \whileyinline{B}.
But there is no obvious algorithm that detects these relations.
A trivial approach is to continuously replace type names with their declarations and compare the resulting structure.
But this will never yield the same result because unfolding does not terminate for these recursive structures.
To solve this challenge, \whiley uses \emph{type automata}.


\subsubsection{Type Automata}\label{section:background-whiley-type-automata}

\begin{wrapfigure}{O}{0.42\linewidth}
% !TeX root = ../../main.tex
\begin{tikzpicture}[->,>=stealth',shorten >=1pt,auto,node distance=2.8cm,semithick,initial text={\whileyinline{LinkedList}},font=\footnotesize]
\tikzstyle{every state}=[draw]

\node[initial above,state] (A)                     {$\cup$};
\node[state]               (B) [below left  of=A]  {\whileyinline{null}};
\node[state]               (C) [below right of=A]  {rec};
\node[state]               (D) [below left  of=C]  {\whileyinline{int}};

\useasboundingbox (current bounding box.south west) rectangle ($(current bounding box.north east)+(.9cm,0)$);

\path (A) edge (B)
          edge (C)
      (C) edge node [swap, pos=0.25] {head} (D)
          edge [out=-45, in=23, looseness=2] node [swap, pos=0.15, xshift=-3mm] {tail} (A);
\end{tikzpicture}

\caption{Automaton for \whileyinline{LinkedList}}
\label{figure:automaton-linkedlist}
\end{wrapfigure}

We give here a concise overview of how types are internally represented in \whiley.
A more extensive introduction can be found in \cite{WYAUTL, pearce2011algorithmic}.

A type is represented using a \emph{finite state automaton}.
Each automaton state describes a type.
The type represented by the automaton itself is the one of its start state, \whiley uses the term \emph{root state}.
Each state is annotated with a \emph{kind} that denotes what class of type that state represents.
Possible kinds are for example the primitive types, union, reference, record, function and method.
Some states have outgoing transitions to other states, called \emph{children}.
A reference state has exactly one child, namely the state describing the referenced type.
States for primitive types do not have any children.
A union state has a child for every type that is part of the union.
States can carry supplementary data.
Functions and methods for example are annotated with the number of parameters.
Records carry the names of their fields and a flag whether the record is open.
\myref{Figure}{figure:automaton-linkedlist} illustrates the type automaton for the type \whileyinline{LinkedList} from the example above.

Values in \whiley always have the form of a finite tree, i.e. they cannot be cyclic.
The recursive type \whileyinline{LinkedList} does not limit the depth of its values, but \whiley does not provide a way to produce a cyclic value or values of infinite depth.
The type automata are designed such that they accept such a tree value if and only if it is contained in the automaton's type.

\whiley implements a single operation on type automata:
given two automata $A_1$ and $A_2$ it is possible to check whether the intersection $A_1 \cap A_2$ is not empty, i.e. if there is a value that is accepted by both $A_1$ and $A_2$.
It is further possible to optionally invert one or both automata.
We use the notation $\overline{A}$ for the inversion of automaton $A$.
Subtyping can be reduced to this intersection problem:
$T_1$ is a subtype of $T_2$ if and only if $A_1 \cap \overline{A_2} = \emptyset$, i.e. the intersection is empty, where $A_1$ and $A_2$ are the type automata representing $T_1$ and $T_2$.
In other words: the values in $T_1$ are a subset of the values in $T_2$.

The implementation of that intersection check is aware of recursive types.
It recursively descends into the children of both given automata until a decision can be made.
For recursive types, the automata will be cyclic.
To break infinite recursion, the algorithm tracks which states have already been compared and assumes that there is an intersection for a pair of states if recursion reaches the same comparison again.


\subsection{Flow Typing}\label{section:background-whiley-flow-typing}

The type of a variable in \whiley changes along the control flow that the program takes.
This technique is called \emph{flow typing}.
Consider the following example:

\begin{whileycode}
function f(int|bool|null x)->(int r):
	if x is null:
		return 0

	if x is bool:
		if x:
			r = 1
		else:
			r = 0

	else:
		r = x

	return r
\end{whileycode}

Function \whileyinline{f} accepts a parameter that can have any of the types \whileyinline{int}, \whileyinline{bool} or \whileyinline{null}.
The type of \whileyinline{x} is the union of these types, as declared in the function signature.
In line~2 we do a so-called \emph{type test}.
Each type test changes the current type of the tested variable.
Inside the true branch \whileyinline{x} will have type \whileyinline{null}.
\whiley sees that the program flow can take two paths: if the condition is true, then we will end up in line~3 and immediately return from our function.
The other path implies that \whileyinline{x} is not \whileyinline{null}.
\whiley calculates a type \whileyinline{int|bool|null - null} which simplifies to \whileyinline{int|bool}.
This is the type of \whileyinline{x} from line~4 on.

The next type test ensures that \whileyinline{x} is typed as \whileyinline{bool} from line~6 to 9.
Therefore we can use it as boolean expression in line~6.
Finally, the else branch starting in line~11 knows that \whileyinline{x} is not \whileyinline{bool}.
The type is \whileyinline{int|bool - bool} which simplifies to \whileyinline{int}.
Therefore we are allowed to assign \whileyinline{x} to the local variable \whileyinline{r} which is typed as \whileyinline{int}.


\subsection{Heap-Allocated Memory and References}\label{subsection:background-whiley-references}

\whiley allows to allocate memory on the heap.
The \whileyinline{new} operator takes an expression as argument.
It allocates the amount of memory needed to store values of the given expression's type.
The expression is then evaluated and the result stored in the allocated space.
A reference to it is returned as result from the \whileyinline{new} operator.
It cannot be used in functions, only methods can allocate memory.

There is no explicit deallocation operator in \whiley.
Heap-allocated memory has to be either garbage collected at runtime or deallocated with operations inserted by the compiler.
The \emph{Whiley to Java compiler} relies on garbage collection provided by the JVM.
The experimental \emph{C} backend does not yet do deallocation of heap-allocated memory.

The following program gives some examples for memory allocations:

\begin{whileycode}
type Point is {int x, int y}

method m():
	&int r1 = new 1
	&bool r2 = new 3 < 4
	Point p = {x: 5, y: 6}
	&Point r3 = new p
\end{whileycode}

To access the memory pointed to by a reference, the \emph{dereference} operator \whileyinline{*} is used.
We can read from and write to references.
References provide a way to produce aliases.
The following program shows some examples:

\begin{whileycode}
method main(whiley.lang.System.Console console):
	&int r1 = new 1
	&int r2 = r1
	console.out.println(*r1) // 1
	console.out.println(*r2) // 1
	*r2 = 2
	console.out.println(*r1) // 2
	console.out.println(*r2) // 2
	r2 = new 3
	console.out.println(*r1) // 2
	console.out.println(*r2) // 3

	&(&int) rr = new r1
	console.out.println(**rr) // 2
\end{whileycode}

\whileyinline{r2} is initially declared as alias for \whileyinline{r1}.
The assignment in line~6 changes the value on the heap, such that both aliases will read the updated value.
In line~9 we set the reference to something else.
This does not affect the value stored on the heap, it only removes the alias we created initially.
\whileyinline{r2} now points to a separate portion of memory, holding the value \whileyinline{3}.

Line~13 and 14 illustrate that it is possible to create references to references.
In order to read the actual value, two dereference operators are necessary.
