% !TeX root = ../main.tex
\chapter{Implementation}\label{chapter:implementation}

This chapter provides a detailed view on the different parts of the implementation.
A broad overview of our design and a discussion among alternative solutions has been given in \myref{chapter}{chapter:design}.

\whiley is an open source project.
The compiler is written in \java and its source code can be found at \url{https://github.com/Whiley/WhileyCompiler}.
The lifetime extension covered by this thesis has already been merged into the \texttt{develop} branch.
A copy of that \texttt{git} repository is also stored on the digital medium attached to this thesis.

The implementation of our lifetime extension comprises a total of 2388 added and 464 removed lines of \java source code, excluding new test cases which add another 549 lines.
Furthermore, the author of this thesis contributed some independent changes to \whiley, mainly in order to fix bugs discovered while developing the lifetime extension.
These changes cover additional 665 added and 96 removed\footnote{This number excludes the 5725 removed lines from \url{https://github.com/Whiley/WhileyCompiler/pull/582}. That change removed an explicit list of JUnit test cases and added a parameterized test that runs for all available test programs.} lines.

\myref{Section}{section:parser} presents how we change the parser to accept the new syntax.
In \myref{section}{section:type-checking}, we extend the type checker to respect lifetimes.
Method calls and lifetime substitution are presented separately in \myref{section}{section:substitution-and-lookup}.
\myref{Section}{section:wyil} shows the changes to the intermediate language.
