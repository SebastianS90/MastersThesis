% !TeX root = ../main.tex
\section{Type Checking}\label{section:type-checking}

The next phase after parsing a \whiley program is type checking.
The type checker ensures that all functions, methods, statements and expressions are well-typed.
A key part of it is \emph{subtyping}.

All types in \whiley are represented as automata, as described in \myref{section}{section:background-whiley-type-automata}.
The implementation for subtyping can be found in class \javainline{wyil.util.type.SubtypeOperator}.
The \javainline{SubtypeOperator} object stores the type automata of the two types that should be tested.
Method \javainline{isSubtype} translates the subtype query into an intersection test:
\texttt{T1} is a subtype of \texttt{T2} if and only if the intersection of \texttt{T1} with the complement of \texttt{T2} is non-empty.

An intersection can either return a result immediately or depend on one or more recursive intersection queries.
The intersection of \whileyinline{int} and \whileyinline{bool} is empty.
Types \whileyinline{&a} and \whileyinline{&b} intersect if and only if types \whileyinline{a} and \whileyinline{b} intersect.
Recursive queries use the same type automata, but a different automaton state as entry point.
The \javainline{isIntersection} method therefore takes the automaton state index for both types as parameter.
Furthermore, it takes for both types a flag denoting whether we consider the complement instead of the type itself.
In the following, we write \whileyinline{!T} to denote the complement of type \whileyinline{T}.

For recursive types, recursive intersection tests can incur an infinite recursion.
To mitigate that problem, the intersection test is split into two methods, \javainline{isIntersection} and \javainline{isIntersectionInner}.
The first one contains logic to break recursion.
When recursion leads to the same test again, then the intersection is assumed to be non-empty.
Method \javainline{isIntersectionInner} contains the actual intersection logic that needs to be extended in order to introduce lifetimes.


\subsection{Lifetime Relation}
The new subtyping rules presented in \myref{section}{section:design-subtyping} use the \emph{outlives} relation among lifetimes.
We introduce a new class \javainline{wyil.util.type.LifetimeRelation} that stores this relation.
Entries are added while recursing the AST in \javainline{wyc.builder.FlowTypeChecker}.
An instance of \javainline{LifetimeRelation} is passed to \javainline{SubtypeOperator} in order to check the \emph{outlives} constraints in subtyping rules.

The \emph{outlives} relation is a \emph{partial order}.
Lifetimes declared in named blocks are ordered such that the lifetimes of outer blocks outlive the lifetimes of inner blocks.
We store the currently declared named blocks in a \javainline{Stack}.
The type checker treats lifetime \whileyinline{this} like a named block:
method bodies are assumed to be enclosed in a block with name \whileyinline{this} such that it is always the outermost block.
Furthermore, we store a \javainline{Set} of declared lifetime parameters.
They outlive all named blocks, because the lifetimes are declared by the current method's caller.
But there is no order within lifetime parameters.


\subsection{Reference Types}

We need to extend the intersection logic for reference types to respect the annotated lifetimes.
One or both types can be inverted for the test, such that we have to consider four cases:

\begin{enumerate}
\item \whileyinline{&a:A} $\cap$ \whileyinline{!(&b:B)}\\
This intersection is empty if and only if \whileyinline{&a:A} is a subtype of \whileyinline{&b:B}.
As described in \myref{section}{section:design-subtyping}, this is the case if
\begin{itemize}
\item lifetime \whileyinline{a} outlives lifetime \whileyinline{b} and
\item \whileyinline{A} and \whileyinline{B} describe the same type, i.e. the intersections \whileyinline{A} $\cap$ \whileyinline{!B} and \whileyinline{B} $\cap$ \whileyinline{!A} are empty.
\end{itemize}

The first condition can be checked using the newly introduced \javainline{LifetimeRelation}.
Remember that our definition of \emph{outlives} is reflexive.
The second one incurs two recursive intersection queries.

\item \whileyinline{!(&a:A)} $\cap$ \whileyinline{&b:B}\\
This case is symmetric to the first one.

\item \whileyinline{&a:A} $\cap$ \whileyinline{&b:B}\\
We have to check whether there is a value in both \whileyinline{&a:A} and \whileyinline{&b:B}.
We claim that this is equivalent to intersecting \whileyinline{&A} and \whileyinline{&B}, i.e. using the lifeitme \whileyinline{*} instead of \whileyinline{a} and \whileyinline{b}.
If there is a value in both \whileyinline{&a:A} and \whileyinline{&b:B}, then we can take that value and change its lifetime to \whileyinline{*}, which outlives all lifetimes.
Due to the subtyping rule for references, the value will then be in both \whileyinline{&A} and \whileyinline{&B}.
For the other direction, assume that there is a value in both \whileyinline{&A} and \whileyinline{&B}.
Then it is also in both \whileyinline{&a:A} and \whileyinline{&b:B} because the lifetime \whileyinline{*} outlives \whileyinline{a} and \whileyinline{b}.


\item \whileyinline{!(&a:A)} $\cap$ \whileyinline{!(&b:B)}
The complement of a reference type contains values of types that are not references, e.g. \whileyinline{true} and \whileyinline{0}.
These values are in the intersection, so it is non-empty regardless of \whileyinline{a}, \whileyinline{b}, \whileyinline{A} and \whileyinline{B}.
\end{enumerate}


\subsection{Method Types}\label{section:implementation-typecheck-methods}

Our extension adds lifetime parameters and context lifetimes to method types.
These concepts have been introduced in sections \ref{section:design-lifetime-parameters} and \ref{section:design-context-lifetimes}.
They have to be considered for intersection tests.

Lifetime parameters are an ordered list of lifetime names that may appear in the methods parameter and return types.
The actual name of these lifetime parameters does not matter for the method's type, they are only placeholders.
Consider the following method types:
\begin{enumerate}
\item \whileyinline{method<a,b>(&a:int)->(&b:int)}
\item \whileyinline{method<b,a>(&b:int)->(&a:int)}
\item \whileyinline{method<a,b>(&b:int)->(&a:int)}
\end{enumerate}

\noindent Our intuition suggests that the first two types are the same, because we only need to rename the lifetime parameters.
But the third type is different:
it cannot be transformed into the first two types only by renaming lifetime parameters.

In \myref{section}{section:substitution-subtyping} we present an algorithm to substitute the actual lifetime names with abstract placeholders.
The first lifetime parameter will be named \whileyinline{$1}, the second one \whileyinline{$2} and so forth.
The types above will therefore be transformed to:
\begin{enumerate}
\item \whileyinline{method<$1,$2>(&$1:int)->(&$2:int)}
\item \whileyinline{method<$1,$2>(&$1:int)->(&$2:int)}
\item \whileyinline{method<$1,$2>(&$2:int)->(&$1:int)}
\end{enumerate}

\noindent Using this substitution and ignoring the symmetric and trivial case, we have to adopt the following intersections:

\begin{enumerate}
\item \whileyinline{method1} $\cap$ \whileyinline{!method2}\\
This intersection is empty if and only if \whileyinline{method1} is a subtype of \whileyinline{method2}.
As described in \myref{section}{section:design-discussion-lambda}, every context lifetime in the subtype must outlive a context lifetime in the supertype.
Furthermore, the number of lifetime parameters must be the same in both method types.

The remaining condition is that parameter types are contra-variant and return types co-variant.
This is not changed by our lifetime extension, except for the substitution of lifetime parameters.

\item \whileyinline{method1} $\cap$ \whileyinline{method2}\\
We have to check whether there is a method that is in both method types.
For such a method to exist, the number of lifetime parameters must match.
But we do not need to consider context lifetimes here:
an intersecting method can be chosen with an empty set of context lifetimes.
Then it is in method types with arbitrary context lifetimes, provided that the remaining conditions are met.

The remaining condition is that return types intersect.
This is not changed by our extension, but we need to apply the lifetime substitution.
\end{enumerate}
