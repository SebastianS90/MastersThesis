% !TeX root = ../main.tex
\chapter{Introduction}

Programmers today have the choice among dozens of different programming languages.
They can be classified in paradigms.
These are for example functional languages (Haskell, Erlang, Racket), procedural languages (C, Fortran, Pascal) or object-oriented languages (Java, C++, Perl).
Programming languages can be compiled (C, C++, Pascal) or interpreted (Perl, Python, Ruby).
Another difference is the level of abstraction regarding memory management:
some languages like C delegate all memory management to the programmer, others use automated techniques to free unused memory.
The first approach, \emph{manual memory} management, comes with some drawbacks:
modern programs are inherently complex and especially when in comes to multi-threaded software, memory management can be quite cumbersome.
While forgetting to free memory in time only leads to memory leaks, a premature deallocation generates dangling pointers.
Accessing such a pointer is referred to as \emph{use after free}.
It results in undefined behavior, as the memory pointed to may already have been reallocated to store something else.

One approach to free programmers from the burden of manual memory management is \emph{garbage collection}.
The runtime environment continuously analyzes which memory is no longer reachable by the running program and frees it accordingly.
One way to implement it is reference counting.
This is known to work well with acyclic structures \cite{Bevan1987}, but more complex analysis is needed if structures are cyclic.
Other algorithms are for example mark-and-sweep and copying collection \cite{Zorn:1990:CMS:91556.91597}.
Garbage collection was made popular by the \java programming language \cite{Venners:1996:IJV:541297}.

All techniques for garbage collection impose some form of runtime overhead.
Depending on the algorithm in use it might also be impossible to guarantee specific reaction times, as garbage collection can nondeterministically interrupt program execution \cite{andreae2007scoped}.
While today's computers get more and more powerful such that garbage collection might not be a problem, technology comes up with small embedded devices like drones or even small computers as part of pacemakers.
Here we need a small and energy efficient runtime environment, but cannot allow for any avoidable programming bugs as they might cause serious incidents with those kinds of devices.
Another field where garbage collection is usually not applicable is operating systems development.

One solution to this trade-off between performance and safety is to statically analyze the program and let the compiler infer all necessary deallocation points.
\rust is a new programming language that employs a notion of ownership and lifetimes.
The programmer is enforced to obey some constraints and the compiler ensures that program execution will be safe, without having an expensive runtime environment.
In fact, \rust programs show a similar performance to \emph{C} programs \cite{RustBenchmarks}.


\section{Whiley}

The way how \java guarantees memory safety is to add some runtime checks.
When accessing an array there will be a check that the index is within the array's bounds.
Another common check is to ensure that a dereferenced pointer is not \texttt{null}.
A \texttt{RuntimeException} gets thrown if one of these checks fails.

An alternative to this approach is \emph{verification}.
Given a program, you should be able to statically prove that each array access or pointer dereference will be safe for every possible program execution.
This technique ensures that these runtime faults cannot occur at all and renders those checks redundant.
Another application is \emph{specification}.
The programmer annotates each function with a mathematical description of what is expected to happen.
Using verification, we then can check that this is actually the case for the given implementation.

Most programming languages do not provide assistance for these tasks.
If at all possible, programmers need to use special extensions or external tools to specify and verify their software.
As a result, casually written programs are often neither verified nor formally specified.
\whiley is a programming language that aims to make specifying and verifying software a common practice \cite{WLS}.
Included directly in the language core, the compiler ships with an integrated verifier that checks all functions and methods against their specification.
Furthermore, it ensures absence of runtime exceptions as described above.
This provides a perfect base for safety critical systems.
However, \whiley currently compiles to byte code for the \emph{Java Virtual Machine} (JVM), which comes with a rather large runtime footprint unsuitable for some kinds of smaller embedded devices.
There have been efforts to compile \whiley to embedded devices, but memory management was a big challenge, because \whiley currently relies on garbage collection done by the JVM \cite{Matt, IntegerRangeAnalysis}.


\section{Contributions}

In this thesis, we extend the \whiley programming language to introduce a concept of lifetimes.
Similar to \rust, each lifetime describes a region in the source code where a specific reference is considered to be alive.
During that timespan, the memory pointed to is guaranteed to still be available.
Afterwards it can safely be freed by the compiler.

We extend the language syntax to support lifetime annotations.
Several internals of the \whiley compiler are affected by our changes, e.g. the type system needs to statically check constraints imposed by lifetimes.

We implement our changes and add several test cases.
The \whiley maintainer accepted our code and released a new version including lifetimes\footnote{\url{http://whiley.org/2016/05/28/whiley-v0-3-40-released/}}.

Our extension can be used for improved memory management.
We describe an approach how to implement automatic and safe deallocation of heap-allocated memory without garbage collection.
This allows to write \whiley compilers targeting embedded devices that are not capable of running a full JVM.


\section{Organization}

\myref{Chapter}{chapter:background} gives a short introduction to the \whiley and \rust programming languages.
We do not cover every detail of these languages, but we establish a basic knowledge of the key features and concepts which are necessary to understand our contribution.
Afterwards we examine other related work.

In \myref{chapter}{chapter:design}, we give a high level view of our lifetime extension.
We first motivate our changes and then present the new syntax and meaning of lifetimes.
Finally, we discuss possible alternatives and why they have been discarded.

\myref{Chapter}{chapter:implementation} covers the actual implementation of our language extension in the \whiley compiler.
We provide an overview of the necessary code changes and elaborate on some difficult parts.

We give an evaluation of our work in \myref{chapter}{chapter:evaluation}.
We analyze the performance impact for checking lifetimes using the test programs shipping with \whiley.
Furthermore, we sketch a possible implementation for memory management using lifetimes to show how the \whiley to \emph{C} compiler and possibly other backends can benefit from our work.

\myref{Chapter}{chapter:conclusion} concludes this thesis and shows options for future work.
