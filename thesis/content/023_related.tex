% !TeX root = ../main.tex
\section{Related Work}\label{section:background-related}

This section provides a rough overview of work related to ownership systems, automatic memory management without garbage collection, and verification.

\subsection{Ownership}
Work on ownership models dates back to 1998.
Clarke et al. \cite{Clarke:1998:OTF:286936.286947} presented a type system featuring ownership types to control aliasing.

Boyapati et al. \cite{Boyapati:2003:OTS:781131.781168} extend the Real-Time Specification for Java (RTSJ) and introduce a static type system using ownership types.
It guarantees that the runtime checks ensuring memory safety in RTSJ will not fail, thus allowing to remove these checks.

In order to allow for parametric ownership types, Potanin et al. \cite{Potanin:2006:GOG:1167473.1167500} developed an ownership system for Java 5, combining both generic types and generic ownership.


\subsection{Region-Based Memory Management}
Grossman et al. developed a programming language called \emph{Cyclone} \cite{cyclone2002}.
It is an extension of \emph{C} that provides safe memory management.
The unsafe \texttt{free} operator is removed.
The memory is divided into several \emph{regions}.
Each region corresponds to a block in the program code.
When the control flow leaves the block, then its region is freed automatically.
Intraprocedural static analysis on region annotations in pointer types ensure memory safety.

Instead of relying on the programmer to annotate the program with regions, \cite{RegionJava} proposes an interprocedural region analysis algorithm that reads a \java program and generates an equivalent program with region-based memory managed support.

These region-based techniques come with the disadvantage that only a whole region can be freed, deallocation of a single object inside the region is not permitted.
Region-based memory management therefore utilizes more memory than garbage collection \cite{RegionJava,PoolAllocation}.
Dhurjati et al. present an interprocedural analysis for \emph{C} programs that manages memory in \emph{pools} \cite{PoolAllocation}.
Explicit \texttt{free} commands are allowed if they do not violate the safety constraints.
Otherwise the command will be removed, but the memory is still freed when the whole pool gets deallocated.


\subsection{Alias Analysis}
Aliasing is a key problem in memory management and verification.
We say that a reference is an alias if there are multiple references to the same memory location.
The referenced memory can be mutated through any of these aliases, which affects also the other references.
Furthermore, deallocating the memory while there are still aliases to it leads to dangling pointers.

\emph{AliasJava} \cite{Aldrich:2002:AAP:582419.582448} is an annotation system for Java.
It extends the type system to explicitly annotate to what extend references can be aliased.
These annotations enable programmers to better reason about aliasing.
Furthermore, the authors present an algorithm that can automatically infer suitable annotations.

Brandauer et al. \cite{Brandauer:2015:DDF:2814270.2814280} recently proposed \emph{Disjointness Domains} as a system for fine-grained alias control.
All references belong to these domains that impose restrictions on aliasing within a single domain and among several domains.


\subsection{Separation Logic}
\emph{Separation logic} \cite{reynolds2002separation} is an extension of \emph{Hoare logic} \cite{Hoare:1969:ABC:363235.363259}.
It adds a way to reason about heap memory.
We can make statements about disjointness of heap regions which basically is the absence of aliasing between both regions.
Furthermore, we can reason about the actual values stored on the heap.

Tuch et al. \cite{Tuch:2007:TBS:1190216.1190234} establish a formal model for a subset of the \emph{C} programming language.
They use the Isabelle/HOL theorem prover to reason about programs with pointers.
A challenge is to model the real behavior of finite integers with overflow semantics.
