% !TeX root = ../main.tex
\section{Lexing and Parsing}\label{section:parser}

The \whiley parser can be found in package \javainline{wyc.io} and consists of two parts:
the \emph{lexer}, implemented in class \javainline{WhileyFileLexer}, transforms the source file into a \emph{token stream}.
Class \javainline{WhileyFileParser} is the actual parser which generates an \emph{abstract syntax tree} (AST) from these tokens.

We had to adjust these components to support the syntax changes presented in \myref{section}{section:syntax}.


\subsection{Abstract Syntax Tree}
The \java classes for the AST nodes can be found in package \javainline{wyc.lang}.
The following additions have been made:
\begin{itemize}
\item \textbf{Declared Method}: a list of lifetime parameters was added to the AST node representing a method.
\item \textbf{Lambda Expression}: a list of lifetime parameters and a set of context lifetimes was added to the AST node representing a lambda expression.
\item \textbf{Method Invocation Expression}: a list of lifetime arguments was added to the AST node representing a method invocation expression and the return type now reflects lifetime substitution as presented in \myref{section}{section:substitution-return-type}.
\item \textbf{Reference Type}: a lifetime name was added to the AST node representing a reference type.
\item \textbf{Method Type}: a list of lifetime parameters and a set of context lifetimes was added to the AST node representing a method type.
\item \textbf{Named Block}: a new AST node for \emph{named blocks} was added. It contains the block's name and the list of statements within that block.
\end{itemize}


\subsection{Lexer}
The only change to the lexer is to add \whileyinline{this} as a new keyword.
That change is backwards-incompatible, because it affects all programs that use \texttt{this} as an identifier.
Some of the existing test cases were affected and had to be updated to use another name instead of \texttt{this}.
This is due to the historic actor model in \whiley that has been removed a long time ago.
It is unlikely that many programs are affected by our change.


\subsection{Parser}
Changes to \javainline{WhileyFileParser} are more complex.
The parser is structured in several methods that are responsible to \emph{consume} tokens for a specific AST node, e.g. an expression, a method declaration or a type.
To chose the right method, a \emph{lookahead} of one or more next tokens is used.

\subsubsection{Enclosing Scope}
While parsing a function or method, some contextual information is tracked in an object of a class called \javainline{EnclosingScope}.
The main purpose of it is to store declared variable names.
We extend that class to also store the declared lifetime names.

\whiley already expects variable names to be unique in a scope, i.e. it is not allowed to declare a local variable where the name is already used for another variable that is still in scope.
We extend this constraint for lifetimes.
Particularly, local variables and lifetimes must have different names.
We use this fact later for method invocations.

We add helper methods in \javainline{EnclosingScope} to declare new variables or lifetimes.
They check whether the name is already in use and throw an appropriate exception if that is the case.

\subsubsection{Methods}
The main entry point for parsing \whiley functions and methods is the parser method \javainline{parseFunctionOrMethodDeclaration}.
We extend it to also parse lifetime parameters for method declarations.
They are added to the \javainline{EnclosingScope} such that the lifetime names can be used in a method's parameter and return types and within the method body.
Lifetime \whileyinline{this}, denoting the body of a method, is added just before parsing the method body, thus ensuring that it cannot be used in parameter or return types.

\subsubsection{Headless Statements}
Some statements are not identifiable with a dedicated keyword.
They are called \emph{headless statements} and handled in method \javainline{parseHeadlessStatement}.
\whiley used to have the following headless statements:
assignments, invocations, and variable declarations.
The latter one starts with a type.
But there are token sequences that can be parsed as a type or expression, for example an identifier can refer to a type alias or a local variable.
The sequence \whileyinline{&foo} can denote the type \emph{reference to foo} or an address expression, where the value is the method with name \whileyinline{foo}.

To disambiguate these cases, the compiler attempts to parse a type and checks whether it cannot be parsed as expression.
In case of such a \emph{definite type}, the headless statement is a variable declaration.
The check is done in method \javainline{mustParseAsType}.

We extend that check to be aware of lifetimes in reference types.
The sequences \whileyinline{&*:foo}, \whileyinline{&this:foo} and \whileyinline{&a:foo} cannot be parsed as expression.
The legacy form \whileyinline{&foo} is syntactic sugar for \whileyinline{&*:foo}.
But the check needs to know whether the default lifetime \whileyinline{*} was explicitly given or omitted.
We therefore also extend the AST node for reference types to store such a flag.

\subsubsection{Named Blocks}
The newly introduced \emph{named blocks} are headless statements.
We therefore extend \javainline{parseHeadlessStatement}.
A named block starts with an identifier followed by a colon and a line break.

\begin{whileycode}
myblock:
	// statements inside the block
\end{whileycode}

When parsing a named block, an independent clone of \javainline{EnclosingScope} is created.
We add the identifier as a lifetime name and parse the statements within the named block.
All following statements with greater indent are considered to be part of the named block.
The logic for parsing that kind of indented blocks already existed, e.g. to parse the body of \whileyinline{while} loops.
We reused that logic for named blocks.

\subsubsection{Method Invocations}
Method invocation expressions can now have optional lifetime arguments.
Our syntax for calling a method with explicit lifetime arguments is \whileyinline{m<a, b>(x, y)}.
But the part \whileyinline{m<a} could also be an expression, where both \whileyinline{m} and \whileyinline{a} are numeric local variables.
To disambiguate, we enforce that lifetime names and local variables are disjoint.
If \whileyinline{a} is a lifetime, then it has to be an invocation.
The new disambiguation logic has been added in \javainline{parseAccessExpression} and \javainline{parseTermExpression}.

\subsubsection{Reference Type and Method Type}
\begin{sloppypar}
Reference types are parsed in \javainline{parseReferenceType} and method types in \javainline{parseFunctionOrMethodType}.
We extend both methods to reflect our syntax extension.
Reference types can be annotated with a lifetime, e.g. \whileyinline{&this:int}.
Method types now include context lifetimes and lifetime parameters, e.g. \whileyinline{method[this]<a,b>(&a:int,&b:int)->(&this:int)}.
\end{sloppypar}

The parser checks that the lifetime name in reference types and the context lifetimes for method types are already declared.
Declared lifetimes are stored in the \javainline{EnclosingScope} object.
We therefore need it as parameter to both methods and all methods that call them either directly or transitively.
Most of the parser methods already have that parameter, but some methods and their invocations had to be adjusted to add it.


\subsubsection{Allocation Expression}
Memory allocation is done using the \whileyinline{new} expression.
Our extension allows to specify a lifetime for the allocated portion of memory:
\whileyinline{this:new 1}.
Allocation expressions are parsed by method \javainline{parseNewExpression}.
We extend it to parse the optional lifetime.

Allocation expressions can now start with different tokens:
either \whileyinline{new} if the lifetime is omitted, or any lifetime (identifier, \whileyinline{this}, or \whileyinline{*}).
There is a lookahead logic in \javainline{parseTermExpression} that decides when to call \javainline{parseNewExpression}.
We extend that logic to also consider a sequence of three tokens:
lifetime, colon, \whileyinline{new}.

\subsubsection{Lambda Expressions}
Lambda expressions are parsed in method \javainline{parseLambdaExpression}.
We extend that method to parse optional context lifetimes and lifetime parameters.
