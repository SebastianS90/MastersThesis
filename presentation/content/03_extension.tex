% !TeX root = ../main.tex
\section[Extension]{Lifetime Extension for Whiley}

\begin{frame}
	\frametitle{Agenda}
	\tableofcontents[currentsection,hideothersubsections]
\end{frame}

\subsection{Goals}
\begin{frame}
	\frametitle{Goals}
	\begin{itemize}[<+->]
		\item Backwards compatibility (as much as possible)
		\item Keep the language simple
		\item Develop a basis for memory management without garbage collection
		\begin{itemize}
			\item Future improvement of the \emph{Whiley to C Compiler} (needed for embedded systems)
		\end{itemize}
	\end{itemize}
\end{frame}

\subsection{Design}
\begin{frame}
	\frametitle{Designing the Extension}
	\begin{itemize}[<+->]
		\item Reference Type annotated with (optional) lifetime: \whileyinline{&a:T}
		\item Allocation operator annotated with (optional) lifetime: \whileyinline{a:new expr}
		\item Possible lifetimes: \whileyinline{this}, \whileyinline{*}, \whileyinline{a}
		\item Named Blocks to declare lifetime names
	\end{itemize}
\end{frame}

\begin{frame}[fragile]
\frametitle{Example}
\begin{whileycode}
method main():
	&this:int x = this:new 1

	a:
		&a:int y = a:new 2
		y = x
\end{whileycode}
\end{frame}

\begin{frame}
	\frametitle{Lifetime Invariant}
	\begin{block}{Invariant}
		An initialized reference of type \whileyinline{&a:T} points to a portion of memory that will be alive at least until the program's control flow leaves the region described by lifetime \whileyinline{a}.
	\end{block}

	\begin{definition}[Liveness]
		A memory location is \emph{alive} if and only if:
		\begin{itemize}
			\item it has been allocated using the \whileyinline{new} operator and
			\item it has not yet been freed by the runtime system
		\end{itemize}
	\end{definition}
\end{frame}

\begin{frame}[fragile]
	\frametitle{Outlives Relation}
	\begin{definition}[Outlives]
		A lifetime \whileyinline{a} outlives lifetime \whileyinline{b} (denoted \whileyinline{a} $\succ$ \whileyinline{b}) if the region described by \whileyinline{b} is fully contained in the region described by \whileyinline{a}.
	\end{definition}
	\pause
	\begin{block}{Example}
		\begin{columns}
			\begin{column}{0.45\linewidth}
\begin{whileycode}
method m():
	a:
		// ...
	b:
		c:
			// ...
\end{whileycode}
			\end{column}
			\begin{column}{0.5\linewidth}
				\pause
				\begin{itemize}
					\item \whileyinline{this} $\succ$ \whileyinline{a}
					\item \whileyinline{this} $\succ$ \whileyinline{b} $\succ$ \whileyinline{c}
				\end{itemize}
			\end{column}
		\end{columns}
	\end{block}
\end{frame}

\subsection{Subtyping}
\begin{frame}[fragile]
	\frametitle{Subtyping}
	\begin{block}{Subtyping of Reference Types}
		A type \whileyinline{&a:A} is subtype of \whileyinline{&b:B} if and only if
		\begin{itemize}
			\item lifetime \whileyinline{a} outlives lifetime \whileyinline{b} and
			\item \whileyinline{A} and \whileyinline{B} describe the same type
		\end{itemize}
	\end{block}
	\pause
	\begin{block}{Why the \emph{same} type?}
\begin{whileycode}
method m():
	&int x = new 5
	&(int|null) y = x
	*y = null
\end{whileycode}
	\end{block}
\end{frame}

\subsection{Lifetime Parameters}
\begin{frame}
	\frametitle{Lifetime Parameters}
	\begin{itemize}
		\item Method with reference types as parameters
		\item What should be the lifetime?
		\item<2> Use parametric lifetimes
	\end{itemize}
\end{frame}

\begin{frame}[fragile]
\frametitle{Lifetime Parameters}
\begin{whileycode}
method <a> m(&a:int x) -> &a:int:
	if ((*x) == 42):
		return x
	else:
		return a:new 42

method main():
	&this:int x = this:new 1
	&this:int y = m<this>(x)
\end{whileycode}
\end{frame}

\subsection{Lifetime Substitution}
\begin{frame}
	\frametitle{Lifetime Substitution}
	\begin{itemize}
		\item Consider the method \whileyinline{method <a> m(&a:int x) -> &a:int:}
		\item Call it with parameter of type \whileyinline{&this:int}
		\item What should be the return type?
		\item<2-> Substitute lifetime parameter \whileyinline{a} with lifetime argument \whileyinline{this}, get \whileyinline{&this:int}.
		\item<3-> Capturing?
	\end{itemize}
\end{frame}

\begin{frame}
	\whileyinline{type mymethod is method<a>(&*:mymethod)->(&a:mymethod)}

	\vspace*{2mm}
	\pause
	\begin{minipage}{0.4\linewidth}
		\begin{tikzpicture}[->,>=stealth',shorten >=1pt,auto,node distance=2.8cm,semithick,initial text={\whileyinline{mymethod}},font=\small]
		\tikzstyle{every state}=[draw]
		
		\node[initial above,state] (A)                     {\whileyinline{method<a>}};
		\node[state]               (B) [below left  of=A]  {\whileyinline{&*:}};
		\node[state]               (C) [below right of=A]  {\whileyinline{&a:}};
		
		\useasboundingbox ($(current bounding box.south west)-(4mm,2mm)$) rectangle ($(current bounding box.north east)+(4mm,0)$);
		
		\path (A) edge node [sloped, anchor=center, above] {param} (B)
		          edge node [sloped, anchor=center, above] {return} (C)
		      (B) edge [out=-135, in=157, looseness=1.2] (A)
		      (C) edge [out=-45, in=23, looseness=1.2] (A);
		\end{tikzpicture}
	\end{minipage}
	\pause
	\begin{minipage}{0.25\linewidth}
		\begin{tikzpicture}[->,>=stealth',shorten >=1pt,auto,node distance=2.2cm,semithick,initial text={},font=\small]
		\tikzstyle{every state}=[draw]
		
		\node[initial above,state] (A)               {\whileyinline{&a:}};
		\node[state]               (B) [below of=A]  {\whileyinline{method<a>}};
		\node[state]               (C) [below of=B]  {\whileyinline{&*:}};
		
		\useasboundingbox ($(current bounding box.south west)-(6mm,2mm)$) rectangle ($(current bounding box.north east)+(7mm,0)$);
		
		\path (A) edge (B)
		      (B) edge node [swap, pos=0.3] {param} (C)
		          edge [out=-45, in=23, looseness=1] node [swap, pos=0.1, xshift=-3mm] {return} (A)
		      (C) edge [out=-135, in=157, looseness=1] (B);
		\end{tikzpicture}
	\end{minipage}
	\pause
	\begin{minipage}{0.3\linewidth}
		\begin{tikzpicture}[->,>=stealth',shorten >=1pt,auto,node distance=2.2cm,semithick,initial text={},font=\footnotesize]
		\tikzstyle{every state}=[draw]
		
		\node[initial above,state] (A)               {\whileyinline{&this:}};
		\node[state]               (B) [below of=A]  {\whileyinline{method<a>}};
		\node[state]               (C) [below left  of=B]  {\whileyinline{&*:}};
		\node[state]               (D) [below right of=B]  {\whileyinline{&a:}};
		
		\useasboundingbox ($(current bounding box.south west)-(4mm,2mm)$) rectangle ($(current bounding box.north east)+(4mm,0)$);
		
		\path (A) edge (B)
		      (B) edge node [pos=0.3] {p} (C)
		          edge node [swap, pos=0.3] {r} (D)
		      (C) edge [out=-135, in=157, looseness=1.2] (B)
		      (D) edge [out=-45, in=23, looseness=1.2] (B);
		\end{tikzpicture}
	\end{minipage}
\end{frame}

\begin{frame}
	\frametitle{Algorithm}
	\begin{itemize}[<+->]
		\item Recursively copy all states (depth-first)
		\item Thereby apply the substitution
		\item When entering a method type, do not substitute the lifetime parameters in subgraph
		\item Reuse already copied state if it ignores the same lifetime parameters
		\item $n$ states and $m$ lifetimes $\Rightarrow$ maximal $n * 2^m$ states in substituted type
	\end{itemize}
\end{frame}
