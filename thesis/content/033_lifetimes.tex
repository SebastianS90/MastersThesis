% !TeX root = ../main.tex
\section{Lifetimes}\label{section:design-lifetimes}

A \emph{lifetime} in our extension to \whiley describes a region in the source code.
Possible regions are method bodies and named blocks.
Each lifetime has a name.
The lifetime for method bodies is named \whileyinline{this}.
The lifetime of named blocks have the block's name.
Furthermore, the special lifetime \whileyinline{*} persists for the entire program.

Each reference type is annotated with a lifetime.
It states when the memory pointed to by that reference should be alive.
More precisely, the following invariant must be maintained by the compiler:

\begin{invariant}\label{invariant:alive}
An initialized reference of type \whileyinline{&a:T} points to a portion of memory that will be alive at least until the program's control flow leaves the region described by lifetime \whileyinline{a}.
\end{invariant}

\begin{definition}\label{definition:alive}
A memory location is alive if and only if:
\begin{itemize}
\item it has been allocated using the \whileyinline{new} operator and
\item it has not yet been freed by the runtime system
\end{itemize}
\end{definition}

\begin{figure}[t]
\begin{whileycodec}{design-lifetimes-example}{Example program to illustrate \myref{invariant}{invariant:alive}}
method main():
	&this:int p
	int x = 1
	p = this:new x
	myblock:
		&myblock:int q = p
		&myblock:int r = myblock:new 2
		p = r // illegal
	x = *p
\end{whileycodec}
\end{figure}

We explain the invariant using the example program from \lstref{design-lifetimes-example}.
Line~2 declares a local variable \whileyinline{p} without initializing it with a value.
This is where the condition \emph{initialized reference} of \myref{invariant}{invariant:alive} comes into play.
Although the local variable has already been declared as reference, it does not yet hold a value, i.e. does not point to anything.
It is \emph{uninitialized}.
The \whiley compiler already ensures \emph{definite assignment}, i.e. we cannot read a variable \whileyinline{v} if there might be a control flow path to that read operation without prior assignment to \whileyinline{v} \cite[pages 71, 82]{WLS}.

We initialize \whileyinline{p} in line~4.
The declared type of \whileyinline{p} is \whileyinline{&this:int}.
Its lifetime is \whileyinline{this}, which denotes the body of method \whileyinline{main}.
\myref{Invariant}{invariant:alive} therefore demands that the memory pointed to by \whileyinline{p} must not be freed before the control flow leaves method \whileyinline{main}.
The \whileyinline{new} operator is annotated with a lifetime.
It tells the compiler that the allocated memory must be alive until the control flow leaves method \whileyinline{main} such that it matches the type's lifetime.

In line~6, we assign \whileyinline{p} to another reference \whileyinline{q}.
The type of \whileyinline{q} is \whileyinline{&myblock:int}.
To satisfy \myref{invariant}{invariant:alive}, the memory pointed to by the assigned reference must not be freed before control flow leaves \whileyinline{myblock}.
On the other hand, \myref{invariant}{invariant:alive} ensures that the memory pointed to by \whileyinline{p} will not be freed before the control flow leaves method \whileyinline{main}.
This holds regardless of what has been assigned to \whileyinline{p}, because its declared type is \whileyinline{&this:int}.
Memory that will not be freed before the control flow leaves \whileyinline{main} will in particular not be freed before the control flow leaves \whileyinline{myblock}, because the first implies the latter.
We can therefore allow this assignment.

There is an assignment in line~8 where we attempt to assign \whileyinline{r} to \whileyinline{p}.
Using the same argumentation, we know that the memory pointed to by \whileyinline{r} will not be freed before the control flow leaves \whileyinline{myblock}.
But \whileyinline{p} must point to memory that cannot be freed before the control flow leaves method \whileyinline{main}.
The implication does not hold in this case.
We cannot allow this assignment because it would break \myref{invariant}{invariant:alive} and the safety of other commands might rely on it.
In fact, we dereference \whileyinline{p} in line~9.
It points to the memory assigned in line~7.
The \whileyinline{new} operator was annotated with lifetime \whileyinline{myblock}, telling the compiler that this specific allocation can be freed when control flow leaves \whileyinline{myblock}.
Allowing the assignment in line~8 could therefore lead to dereferencing freed memory, which is unsafe.

We can distinguish the allowed from the forbidden assignment by looking at the involved lifetimes.
In the first case, the lifetime of the assigned value is longer than demanded by the declared type of the variable that we assigned to.
We say that lifetime \whileyinline{this} \emph{outlives} lifetime \whileyinline{myblock}.

\begin{definition}\label{definition:outlives}
A lifetime \whileyinline{a} outlives lifetime \whileyinline{b} if the region described by \whileyinline{b} is fully contained in the region described by \whileyinline{a}.
\end{definition}

\noindent Note that \myref{definition}{definition:outlives} is reflexive, i.e. a lifetime always outlives itself.
Furthermore, the lifetime \whileyinline{*} outlives all lifetimes.


\subsection{Subtyping}\label{section:design-subtyping}

Whether or not assigning a value to a variable is allowed depends on their types.
\whiley allows to assign a value \whileyinline{x} to a variable \whileyinline{v} if and only if the type of \whileyinline{x} is a \emph{subtype} of \whileyinline{v}'s declared type \cite[p. 40]{WLS}.
We therefore need to extend subtyping for references to compare their lifetimes.

\begin{definition}\label{definition:reference-subtype}
A type \whileyinline{&a:A} is subtype of \whileyinline{&b:B} if and only if
\begin{itemize}
\item lifetime \whileyinline{a} outlives lifetime \whileyinline{b} and
\item \whileyinline{A} and \whileyinline{B} describe the same type
\end{itemize}
\end{definition}

\noindent The first condition is part of our extension, the second condition is \whiley's existing rule for subtyping of references.
We cannot allow \whileyinline{A} to only be a subtype of \whileyinline{B} here because the following snippet breaks memory safety:%
\footnote{\whiley actually used to allow it, but the bug was found and resolved while working on this thesis. See \url{https://github.com/Whiley/WhileyCompiler/issues/583}}

\begin{whileycode}
method m(&any x):
	*x = 5 // allowed because x points to 'any' type

method main():
	&bool p = new true
	m(p) // allowed if &bool is a subtype of &any
	bool b = *p // saves '5' into a bool variable
\end{whileycode}

Types in \whiley describe sets of values.
Type \whileyinline{T1} shall be a subtype of \whileyinline{T2} if the set of values in \whileyinline{T1} is a subset of values in \whileyinline{T2} \cite[p. 37]{WLS}.

Our extension complies with the intended semantics, though the set analogy is complicated for references.
Assume \whileyinline{A} and \whileyinline{B} describe the same type and \whileyinline{a} outlives \whileyinline{b}.
Let \whileyinline{p} be a reference of type \whileyinline{&a:A}.
Using \myref{invariant}{invariant:alive}, we know that \whileyinline{p} points to memory that will be alive at least until the control flow leaves \whileyinline{a}.
Since \whileyinline{a} outlives \whileyinline{b}, the memory will in particular be alive until the control flow leaves \whileyinline{b}.
Reference \whileyinline{p} is therefore also in type \whileyinline{&b:B}.


\subsection{Lifetime Parameters}\label{section:design-lifetime-parameters}

Methods can accept references as their parameters.
A method can also return references.
As with every reference type, parameter and return types need a lifetime.
Furthermore, it might be necessary to relate lifetimes of parameter and return types.
\emph{Lifetime parameters} provide a way to declare lifetime names that describe lifetimes originating from the method's caller.
The caller will pass actual \emph{lifetime arguments} for these parameters.

We do a \emph{lifetime substitution} for a method call:
Lifetime parameters are mapped to the provided lifetime arguments.
Every occurring lifetime in the method's parameter and return types is substituted by its appropriate lifetime argument.
Consider the following example:

\begin{whileycode}
method <a> m(&a:int x) -> &a:int:
	if ((*x) == 42):
		return x
	else:
		return a:new 42

method main():
	&this:int x = new 1
	&this:int y = m<this>(x)
\end{whileycode}

The program above declares a method \whileyinline{m} that accepts and returns type \whileyinline{&a:int}.
Lifetime \whileyinline{a} is a lifetime parameter.
The method therefore returns a reference of the same lifetime than given as parameter.
When calling \whileyinline{m}, we give \whileyinline{this} as lifetime argument.
Lifetime \whileyinline{a} in the parameter and return type is therefore substituted with \whileyinline{this}, such that the method invocation returns \whileyinline{&this:int} where \whileyinline{this} refers to \whileyinline{main}'s method body.


\subsection{Context Lifetimes}\label{section:design-context-lifetimes}

Anonymous methods, so-called \emph{lambda methods}, can access local variables from the enclosing method.
This also holds for references.
If such a reference is dereferenced inside the lambda method, then we have to ensure that the referenced memory location is still alive when the lambda method is executed.

To solve this problem, we extend lambda methods and method types with a concept called \emph{context lifetimes}.
These are lifetimes from the enclosing method that can be dereferenced inside a lambda method.
A detailed discussion on the problem and our solution is given in \myref{section}{section:design-discussion-lambda}.
