% !TeX root = ../main.tex
\section{Motivation}\label{section:motivation}
The main objective of our lifetime extension for \whiley is to create a basis for improved memory management.
Consider the following perfectly fine \whiley code:

\begin{whileycodec}{design-motivation-example}{\whiley example without lifetimes}
method inc(&int x) -> &int:
	return new ((*x) + 1)

method fourtytwo() -> &int:
	&int i = new 41
	i = inc(i)
	return i

method main():
	&int p = fourtytwo()
	int i = *p
	assert i == 42
\end{whileycodec}

We have three methods: \whileyinline{inc} is supposed to take a reference to a number and returns a reference to a number whose value is the input increased by one.
Method \whileyinline{fourtytwo} should return a reference to a memory location holding the number \whileyinline{42} and \whileyinline{main} just checks that this is actually the case.

It might not be that useful to use references in this small example program.
But a bigger real world example could actually benefit from passing references instead of values:
passing values can be expensive, if a lot of data has to be copied for each method call.
Furthermore, it might me necessary for the program logic to have call-by-reference semantics.

Before we can discuss compiling this specific program we have to outline typical memory management done in compilers: memory can be allocated either on the stack or on the heap.
Stack-allocated memory will be freed automatically as its containing stack frame gets deallocated when the method returns \cite{DBLP:books/daglib/0030106}.
Data there cannot outlive the current method invocation.
On the other hand, heap-allocated memory has to be explicitly freed by other means, e.g. garbage collection at runtime or manual memory management done by the programmer.
To avoid this kind of overhead, compiler writers prefer to allocate memory on the stack whenever possible.
This usually holds for local variables.
Fixed-size objects that do not escape the method they are allocated in can also be placed on the stack \cite{Choi:1999:EAJ:320384.320386}.

Looking again at our example program from \lstref{design-motivation-example}, we can see that the allocation from line~5 does not leave the scope of method \whileyinline{fourtytwo}, as the reference returned in line~7 is the one allocated by \whileyinline{inc} in line~2.
It might therefore be a good idea to allocate it on the stack.
However, consider the following modified implementation of \whileyinline{inc}:

\begin{whileycodec}{design-motivation-modified-inc}{Modified implementation of \whileyinline{inc}}
method inc(&int x) -> &int:
	*x = (*x) + 1
	return *x
\end{whileycodec}

From a method-local point of view within \whileyinline{fourtytwo} we cannot know what method \whileyinline{inc} will be doing with the passed reference.
While it is totally safe to do the initial allocation on the stack for the program in \lstref{design-motivation-example}, it is not possible for the modified version in \lstref{design-motivation-modified-inc}.

In a real-world example both methods could be part of different program packages.
Only expensive inter-procedural static analysis can find out whether stack-allocation is possible.
Utilizing these kind of analysis comes with the disadvantage that you cannot compile single units any more: Changing the implementation of \whileyinline{inc} from \lstref{design-motivation-example} to \lstref{design-motivation-modified-inc} does not change its signature, but turns compiled code for method \whileyinline{fourtytwo} unsafe if it uses stack-allocation.

Our system of lifetimes in \whiley solves that problem by encoding more behavior into method signatures.
The first example can be annotated with lifetimes as follows:

\begin{whileycode}
method <a> inc(&a:int x) -> &*:int:
	return new ((*x) + 1)

method fourtytwo() -> &*:int:
	&this:int i = this:new 41
	&*:int i2 = inc(i)
	return i2
\end{whileycode}


The signature of method \whileyinline{inc} states that it accepts one parameter which must be a reference to an integer, where the reference can have any lifetime \whileyinline{a}.
Lifetime \whileyinline{a} is a \emph{lifetime parameter} for method \whileyinline{inc}.
Furthermore, the return type is declared to be a reference with the \emph{default lifetime} \whileyinline{*}.

Method \whileyinline{fourtytwo} can now do its allocation using the special lifetime \whileyinline{this} which is valid for the whole method body.
Calling \whileyinline{inc} with type \whileyinline{&this:int} instantiates lifetime parameter \whileyinline{a} with \whileyinline{this}.
The return type is \whileyinline{&*:int}.
We store it to a separate variable that can then be returned from method \whileyinline{fourtytwo}.

When we try to annotate the modified version from \lstref{design-motivation-modified-inc} with lifetimes, we end up with \whileyinline{method <a> inc(&a:int x) -> &a:int} as method signature.
We cannot declare the return type as \whileyinline{&*:int} as the return expression's type is \whileyinline{&a:int} and our system does not allow that one to be a subtype of \whileyinline{&*:int}.
If we try to use a method \whileyinline{inc} with the modified signature, it no longer returns type \whileyinline{&*:int}.
Instead, the lifetime parameter \whileyinline{a} gets substituted with the actual lifetime given as method argument, which in our case is \whileyinline{this}.
Again, method \whileyinline{fourtytwo} cannot return a value of that type, as the declared return type is \whileyinline{&*:int}.

It is important to see that our extension allows to analyze both methods independently.
The decision whether it is safe to allocate memory in \whileyinline{fourtytwo} on the stack instead of using the heap can be made only considering the own method body as well as all signatures of called methods.
Their actual implementation does not matter.
