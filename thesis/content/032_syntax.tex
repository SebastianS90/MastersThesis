% !TeX root = ../main.tex
\section{Syntax}\label{section:syntax}
This section states syntax extensions we make to the \whiley language.
Only updates to the grammar are given, the full specification is available in \cite{WLS}.
An introduction to the \whiley language is provided in \myref{section}{section:background-whiley}.
The syntax is given in a form derived from the popular \emph{Backus–Naur Form} \cite{Knuth:1964:BNF:355588.365140}.
Optional elements are enclosed in square brackets \texttt{[ ]}.
Groups are enclosed in parentheses \texttt{( )}.
Repetitions are denoted by a superscript star.
Nonterminals are presented as plain words and terminals (tokens) are contained in a box.

The core extension is to annotate reference types with a concept of lifetimes:
\begin{syntax}
\syntaxrule{ReferenceType}{\token{\&} \nonterminal{Type}}
\syntaxalternative{\token{\&} \nonterminal{Lifetime} \token{:} \nonterminal{Type}}

\syntaxrule{Lifetime}{\token{*} \alternative \token{this} \alternative \nonterminal{Ident}}
\end{syntax}
\noindent The first option for \nonterminal{ReferenceType} is from the current \whiley specification.
Type \whileyinline{&int} represents a reference to \whileyinline{int} and is equivalent to \whileyinline{&*:int}, where \whileyinline{*} explicitly specifies the \emph{default lifetime}.
It is meant to be used when the programmer cannot or does not want to specify any lifetime.
The compiler then has to manage memory for that reference by some form of static analysis or garbage collection.
Form \whileyinline{&this:int} uses the special lifetime name \whileyinline{this} that denotes the lifetime of the current method body.
We introduce \whileyinline{this} as new keyword to the \whiley language.
Finally, any declared lifetime identifier \whileyinline{a} can be used in the form \whileyinline{&a:int}.

Allocation of memory in \whiley is done using the \whileyinline{new} operator.
We extend the syntax such that we can optionally define a lifetime for the allocated portion of memory:
\begin{syntax}
\syntaxrule{NewExpr}{\token{new} \nonterminal{Expr}}
\syntaxalternative{\nonterminal{Lifetime} \token{:} \token{new} \nonterminal{Expr}}
\end{syntax}
\noindent While \whileyinline{new 42} and \whileyinline{*:new 42} allocate an integer with default lifetime, the new options are \whileyinline{this:new 42} and \whileyinline{a:new 42}, which allocate memory for the annotated lifetime.

To declare new lifetime names we introduce \emph{named blocks}.
They can occur in method bodies and may be nested with other blocks:

\begin{whileycode}
method m():
	int x = 1
	myblock: // declares a new lifetime 'myblock'
		&myblock:int y = new x
		int i = 0
		while (i < 5):
			i = i + 1
			mynestedblock:
				&mynestedblock:int z = new i
				// ...
	// here neither 'myblock' nor 'mynestedblock' are valid
\end{whileycode}

\noindent Formally, we define an additional statement \nonterminal{NamedBlock}:
\begin{syntax}
\syntaxrule{NamedBlock$^\ell$}{\nonterminal{Ident} \token{:} \nonterminal{Block$^\gamma$}}
\syntaxcondition{$\ell < \gamma$}
\end{syntax}
\noindent A \nonterminal{Block} is a list of statements.
The annotated condition $\ell < \gamma$ states that all statements whose indentation is greater than the \nonterminal{Ident}'s indentation are considered as part of the \nonterminal{Block}, i.e. line~4 to 10 in the example program above belong to the \nonterminal{Block} of \whileyinline{myblock}.
Keyword \token{this} is not allowed as identifier for lifetime names in named blocks.

Furthermore, we extend the syntax of method declarations to allow for specifying lifetime parameters.
Function declarations do not need to be extended, as functions are pure and must not make use of references.
Inspired by the syntax for \java generic type parameters, we declare all lifetime names used in method parameters and return type into angle brackets before the method name:

\begin{whileycode}
method <a, b> m1(&a:int x, &b:int y) -> &a:int:
method <a> m2(&{&a:int val} container) -> &a:int:
\end{whileycode}

\noindent The relevant update to the grammar is as follows:
\begin{syntax}
\syntaxrule{MethodDecl}{\token{method} \optionstart \token{<} \nonterminal{LifetimeParameters} \token{>} \optionend}
\syntaxcontinue{\nonterminal{Ident} \token{(} \nonterminal{Parameters} \token{)} \ldots}
\syntaxrule{LifetimeParameters}{\optionstart \nonterminal{Ident} \groupstart\token{,} \nonterminal{Ident}\groupendstar \optionend}
\end{syntax}
\noindent The omitted part contains elements not affected by our lifetime extension, namely return type, \whileyinline{requires} and \whileyinline{ensures} declarations, and the method body.
As seen before in named blocks, the special lifetime \whileyinline{this} is considered to be a keyword and must not appear in \nonterminal{LifetimeParameters}.

Our next syntax extension affects method invocations.
It is needed to pass lifetime names into the \nonterminal{LifetimeParameters} that we just defined.
In line~2 of the following program we call a method \whileyinline{min} with two lifetime parameters.
Its actual implementation is not relevant for this extension, but given as an example:
\begin{whileycode}
method main():
	&this:int x = min<this, this>(this:new 3, this:new 5)

method <a, b> min(&a:int x, &b:int y) -> &a:int|&b:int:
	if (*x) < (*y):
		return x
	else:
		return y
\end{whileycode}

\noindent The updated grammar for method invocation is:
\begin{syntax}
\syntaxrule{InvokeExpr}{\nonterminal{Name} \optionstart \token{<} \nonterminal{LifetimeArgsList} \token{>} \optionend}
\syntaxcontinue{\token{(} \nonterminal{ArgsList} \token{)}}
\syntaxrule{LifetimeArgsList}{\optionstart \nonterminal{Lifetime} \groupstart\token{,} \nonterminal{Lifetime}\groupendstar \optionend}
\end{syntax}
\noindent Readers familiar with the \java syntax for invoking methods with generic type parameters might be concerned about the defined syntax.
A detailed discussion is given in \myref{section}{section:parser}.

As \whiley combines functional and imperative features, we can use methods as values.
To do so, there is a syntactical representation for method types.
Furthermore, methods can be declared as an anonymous method, using so-called \emph{Lambda expressions}.
We have to update both concepts to allow for specifying lifetime parameters.

The following program utilizes our updated syntax.
It declares a type alias \whileyinline{mymethod} for methods that take two references of possibly different lifetimes and yields a reference with default lifetime.
The lambda expression in line~4 represents such a method.
We call it in line~5, using the already introduced syntax for specifying lifetime arguments.

\begin{whileycode}
type mymethod is method<a, b>(&a:int, &b:int)->(&*:int)

method main():
	mymethod m = &<a, b>(&a:int x, &b:int y -> new (*x) + (*y))
	&int p = m<this, *>(this:new 1, new 2)
	assert (*p) == 3
\end{whileycode}

The updated grammar also introduces a concept named \emph{context lifetimes}.
These are lifetimes from the enclosing method that can be used inside the lambda expression.
The example above does not use that concept.
A detailed explanation will be given in \myref{section}{section:design-discussion-lambda}.
The updated grammar is as follows:

\begin{syntax}
\syntaxrule{MethodType}{\token{method} \optionstart \token{[} \nonterminal{ContextLifetimes} \token{]} \optionend}
\syntaxcontinue{\optionstart \token{<} \nonterminal{LifetimeParameters} \token{>} \optionend}
\syntaxcontinue{\nonterminal{ParameterTypes} \optionstart \token{->} \nonterminal{ParameterTypes} \optionend}

\syntaxrule{ContextLifetimes}{\optionstart \nonterminal{ContextLifetime} \groupstart\token{,} \nonterminal{ContextLifetime}\groupendstar \optionend}

\syntaxrule{ContextLifetime}{\token{this} \alternative \nonterminal{Ident}}

\syntaxrule{ParameterTypes}{\token{(} \optionstart \nonterminal{Type} \groupstart\token{,} \nonterminal{Type}\groupendstar \optionend \token{)}}

\syntaxrule{LambdaExpr}{\token{\&} \optionstart \token{[} \nonterminal{ContextLifetimes} \token{]} \optionend}
\syntaxcontinue{\optionstart \token{<} \nonterminal{LifetimeParameters} \token{>} \optionend \token{(}}
\syntaxcontinue{\optionstart \nonterminal{Type} \nonterminal{Ident} \groupstart\token{,} \nonterminal{Type} \nonterminal{Ident}\groupendstar \optionend} \syntaxcontinue{\token{->} \nonterminal{Expr} \token{)}}
\end{syntax}
