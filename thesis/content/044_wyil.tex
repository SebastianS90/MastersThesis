% !TeX root = ../main.tex
\section{Intermediate Language}\label{section:wyil}

The \emph{Whiley Intermediate Language (WyIL)} is a byte-code format.
The \whiley compiler first translates the source file to the intermediate language.
This step involves parsing, type-checking and method lookup.
The intermediate code is then translated to a destination language.
This step is done by a backend, e.g. the \emph{Whiley to Java Compiler}.

We needed to adjust some parts of the intermediate language for our lifetime extension.
The main part is to extend types.
Class \javainline{wyil.lang.Type} contains logic to print types in a human-readable format, to save types in a byte-code format and to interpret that byte-code format to construct the type again.
The affected types are reference types and method types.

The type automata nodes are annotated with supplementary data, e.g. the field names in records types.
There is an \javainline{Object} typed field \javainline{data} for that purpose.
Reference types do not yet use that field.
We therefore can store the lifetime name in it.

Method types are more complicated.
They use the \javainline{data} field to store the number of parameters.
We define a dedicated \javainline{Data} class for method types, containing the number of parameters, the context lifetimes, and the lifetime parameters.
\whiley uses an algorithm to \emph{normalize} type automata.
It relays on suitable \javainline{equals} and \javainline{hashCode} methods and uses an order among types.
To define that order, we need to implement \javainline{Comparable<Data>} in our \javainline{Data} class.
