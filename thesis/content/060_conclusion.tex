% !TeX root = ../main.tex
\chapter{Conclusion}\label{chapter:conclusion}

This chapter briefly revisits the contributions from this thesis.
Furthermore, we provide some options for future extensions of \whiley.
We outline how these extensions are affected by and possibly can benefit from our work on lifetimes.
Afterwards, we draw our final conclusions.


\section{Contribution}
\begin{itemize}
\item We extended the \whiley language and introduced the concept of lifetimes.
Our extension fits to \whiley's environment.
We had to consider structural typing and flow typing, both features are not present in \rust.
We kept the \whiley language as simple as possible and mostly backwards-compatibility, with keyword \whileyinline{this} being the only breaking change.
We gave a discussion on our design decisions and found a solution that can be implemented as part of this thesis.

\item We presented all changes to the \whiley syntax.
We gave formal specifications for each change such that they can be easily incorporated into the \emph{Whiley Language Specification}.
The given examples help especially those readers who are unfamiliar with \whiley to understand each change.

\item We implemented the lifetime extension for the \whiley compiler which is written in \java.
Our implementation covers all aspects that are presented in this thesis.
We can now compile and run \whiley programs with lifetimes and the compiler statically checks all defined constraints.
Main challenges were the substitution algorithm for subtyping, the algorithm for method lookup, and the handling of lambda expressions.
Our implementation has been merged and released as part of \whiley v0.3.40\footnote{\url{http://whiley.org/2016/05/28/whiley-v0-3-40-released/}}.

\item We documented the implementation in \myref{chapter}{chapter:implementation}.
Example programs, figures and further explanation aim to improve understandability of the algorithms in our implementation.

\item We outlined a possible future extension for the \whiley compiler backends that utilizes lifetime information for automatic and safe memory management.
The presented technique can be used to greatly improve the \emph{Whiley to C compiler} (though actually doing this remains as future work).

\item Our implementation ships with several additional test cases, covering valid and invalid programs.
They demonstrate expressiveness of our lifetime extension and are a good indication of correctness of our implementation.
\end{itemize}


\section{Future Work}
\subsection*{Embedded Devices and Memory Management}
One main reason for porting the idea of lifetimes from \rust to \whiley was to provide a basis for automatic and safe memory management without garbage collection.
Previous work on compiling \whiley programs to embedded devices could not deallocate dynamically allocated memory \cite{Matt}.
In \myref{section}{section:evaluation-memory-management}, we presented a technique how a \whiley compiler backend can use lifetime information for this purpose.


\subsection*{Parametric Type Declarations}
There are two main reasons for type declarations in \whiley:
they shorten programs because we can use the type's name instead of its possibly long definition, and more importantly, they allow to define recursive types.
There is currently no way to declare types with parametric lifetimes.
For example, it might be desirable to declare a type for linked lists where all components are connected using references of the same lifetime:

\begin{whileycode}
type LinkedList<a> is null | &a:{int: head, LinkedList tail}
method <a> add(LinkedList<a> l, int x):
	// should modify l and append x
method main:
	LinkedList<this> l = this:new {head: 2, null}
	add(l, 4)
\end{whileycode}

The syntax \whileyinline{LinkedList<a>} is not yet supported.
A future extension could add support for parametric lifetimes together with parametric types, i.e. allowing to define one \whileyinline{LinkedList} that can be instantiated for \whileyinline{int} and \whileyinline{bool} types as \whileyinline{head}.


\subsection*{Inference}
Some programming languages allow to declare variables without specifying the type explicitly, the type will then be inferred from the remaining program.
\rust is one of these languages.
Type inference can speed up programming, because there is less code that has to be written.
On the other hand, explicitly specified types can improve understandability of programs.

Our implementation is able to infer suitable lifetime arguments for method invocations as part of the method lookup algorithm.
But we currently require the programmer to specify lifetimes on reference types and when using the \whileyinline{new} operator.
If no lifetime is given, then the default lifetime \whileyinline{*} is assumed.
It might be desirable for some places to omit an explicit lifetime annotation and automatically infer a better solution.

To benefit from the memory management strategy outlined in \myref{section}{section:evaluation-memory-management}, we should avoid \whileyinline{*} and use smaller lifetimes when possible.
First of all, the lifetime in a \whileyinline{new} expression can be chosen as the smallest lifetime in scope such that the expression still meets the expected type.
An assignment like \whileyinline{&this:int x = new 1} should allocate an integer of lifetime \whileyinline{this} and not \whileyinline{*}.
But inferring the expected type might not be possible if the expression is an argument for a method call:
with overloading, the method to chose and therefore the expected parameter types depend on the provided argument types.
Furthermore, lifetime parameters without lifetime arguments might prevent such a type inference for the method arguments.

For reference types without specified lifetime we currently assume lifetime \whileyinline{*}.
In some cases it is possible to change these lifetimes to \whileyinline{this}, in particular when the method's parameter and return types do not contain any references.

\begin{sloppypar}
Another possible optimization affects method signatures:
a method declared as \whileyinline{method m(&int x, &int y) -> &int} accepts two references of type \whileyinline{&*:int} and returns such a reference.
Depending on the method body, it might be possible to change the signature to \whileyinline{method <a> m(&a:int x, &a:int y) -> &a:int}.
This change does not affect the caller:
the compiler is able to find a suitable lifetime argument for \whileyinline{a}.
It can be \whileyinline{*}, then the change does not have any effect.
But the compiler might also be able to type-check the method call with smaller lifetimes for \whileyinline{a}, e.g. the caller's method body denoted by \whileyinline{this}.
\end{sloppypar}


\subsection*{Concurrency}
As of the time of writing, \whiley does not yet provide concurrency.
The challenge is to support concurrency in a safe way such that data races and runtime faults are avoided, and that it is still possible to verify programs easily.

Verifying programs with shared memory is non-trivial.
The safe core of the \rust language uses communication channels instead of shared memory.
\rust additionally provides some unsafe operations.
They are used in libraries that expose a safe interface for accessing shared memory only via mutual exclusion locks.

We can reason about shared memory using \emph{Concurrent Separation Logic} as proposed by O’Hearn \cite{o2007resources}.
The main challenge here is to find suitable partitions of the heap to establish a proof.
It seems to be difficult to find an algorithm that automatically verifies programs without a lot of work done by the programmer.


\section{Conclusions}
The goal of this thesis was to design and implement a concept of lifetimes for \whiley.
This has been achieved and our added test cases demonstrate that the implementation is actually working.

Furthermore, we hoped that a lifetime system can optimize memory management and in particular remove the dependency on garbage collection.
As shown in the evaluation chapter, our lifetime extension successfully satisfies that goal.
