% !TeX root = ../main.tex
\section{Memory Management}\label{section:evaluation-memory-management}

One goal of introducing lifetimes in \whiley is to provide a basis for improving memory management.
In particular, we want to allow for deallocating dynamically allocated memory without garbage collection.

We establish here a short study on how our lifetime extension can be used for memory management.
This is especially interesting for the \whiley compiler backend that generates \emph{C} code because it currently cannot deallocate dynamically allocated memory.
Some parts of the presented work can also be implemented in the \emph{Whiley to Java Compiler}, but the crucial step of deallocating memory is not possible in the JVM because it relies on garbage collection.

The tactic presented in the following avoids garbage collection.
Memory that is allocated with a lifetime other than \whileyinline{*} will be deallocated automatically.
For memory allocated with \whileyinline{*}, we rely on reference counting and accept that cyclic structures cannot be freed without manually breaking the cycles.
This also holds for \rust \cite{RustCycles}.

A program's memory is divided into the \emph{stack} and the \emph{heap}.
The stack contains local variables and management information to organize method calls.
The heap contains dynamically managed data.
Data on the stack is addressed as a statically known offset to the current stack pointer.
This imposes the constraint that the size of these objects must be known at compile-time.
This holds for primitive types like booleans or 32-bit integers.
Also arrays of these types with a known length have a known size, but we cannot calculate the size of an array with unknown length.

Consider the following program:

\begin{whileycode}
type u8 is int x where 0 <= x && x < 256
method main():
	&this:u8 x = this:new 1
	m(x) // method<a>(&a:u8)
	// ...
\end{whileycode}

\begin{wrapfigure}{o}{0.3\linewidth}
% !TeX root = ../../main.tex
\begin{tikzpicture}[scale=.8]
	\stacktop{}
	\separator
	\cell{\texttt{1}} \coordinate (val) at (currentcell.west);
	\separator
	\cell{} \cellcomL{\texttt{x:}} \coordinate (x) at (currentcell.east);
	\separator
	\stackbottom{}

	\draw[->] (x) -- (val);
\end{tikzpicture}

\caption{Stack Layout}
\label{figure:memory-layout}
\end{wrapfigure}

\noindent We define a type \whileyinline{u8} for 8 bit unsigned integers, because \whiley's type \whileyinline{int} is not bounded in size.
We allocate an integer and pass its pointer \whileyinline{x} to a method \whileyinline{m}.
That method can then modify the referenced value and \whileyinline{main} can see the changes afterwards.
We can allocate the integer directly on the stack.
\myref{Figure}{figure:memory-layout} sketches a possible stack layout for method \whileyinline{main}.
\whileyinline{x} actually contains the memory address of the cell pointed to by the arrow.
Method \whileyinline{m} expects a reference as parameter, i.e. a memory address.
It does not need to know whether the address is part of the stack (as in the figure) or part of the heap.
There is no need to explicitly free the memory allocated in line~3, because the whole stack frame will be freed when \whileyinline{main} returns.

If the allocation is inside a loop and the loop's body does not outlive the lifetime of the allocation, then it might not be possible to place these allocations on the stack:
each loop iteration allocates a distinct portion of memory and with an unknown number of iteration it might not be possible to statically compute a stack layout with space for every allocation.
These allocations and allocations of objects with unknown size must be placed on the heap.

But also the heap-allocated memory can be deallocated without garbage collection:
for each lifetime we manage a list of memory allocations.
When the lifetime goes out of scope, i.e. at the end of a named block or a method, then the allocations in that list will be freed.
Each allocation with \whileyinline{new} generates an entry in that list.
Consider the following sketch of a program:

\begin{whileycode}
type unknownSize is ...
method main():
	&this:unknownSize x = this:new ...
\end{whileycode}

\noindent The memory layout can be as follows:

\begin{figure}[!h]
\centering
% !TeX root = ../../main.tex
\begin{tikzpicture}[scale=.8]

  \stacktop{}
  \separator
  \cell{}   \cellcomL{\texttt{allocations:}} \coordinate (A) at (currentcell.east);
  \separator
  \cell{}   \cellcomL{\texttt{x:}} \coordinate (B) at (currentcell.east);
  \separator
  \stackbottom{}
  \cell[draw=none]{Stack}

  \drawstruct{(5,1)}
  \structcell{\texttt{length = 1}} \coordinate (C) at (currentcell.north west);
  \structcell{} \coordinate (D) at (currentcell.east);

  \drawstruct{(8,-2)}
  \structcell{\textit{variable object}} \coordinate (E) at (currentcell.north west);

  \draw[->] (A) -- ($(C)-(1mm,1mm)$);
  \draw[->] (B) -- ($(E)-(0,1mm)$);
  \draw[->] (D) -- ($(E)+(2mm,2mm)$);

  \draw (6,-4) node {Heap};

\end{tikzpicture}

\end{figure}

The address of the allocation done in line~3 will be stored in the dynamic \texttt{allocations} list.
Assigning another reference to \whileyinline{x} later in the program does not affect the reference in that list.
When method \whileyinline{main} ends then we free all memory from the \texttt{allocations} list and the list itself.
The list is placed on the heap such that we can dynamically add elements, this might be necessary for loops.

When calling a method with lifetime parameters, then the compiler will also pass these \texttt{allocation} lists for all argument lifetimes.
That way, allocations in the callee that use a lifetime parameter can be linked in the appropriate \texttt{allocations} list.

Only lifetimes that are used with \whileyinline{new} or passed as lifetime argument to another method need an \texttt{allocations} list.
We can omit the list for other lifetimes.
Lifetime arguments, either explicitly given or automatically inferred, are available to the called method as lifetime parameters.
The called method might allocate memory for those lifetimes and therefore needs access to their \texttt{allocations} lists.

Furthermore, if the number of allocations for a lifetime is statically known, then we can place the \texttt{allocations} list for that lifetime as an array directly onto the stack.
