% !TeX root = ../main.tex
\section{Test Cases}

\whiley ships with a collection of test cases in the form of programs.
There are valid and invalid programs.
The test cases check that valid programs compile successfully and their execution passes all \whileyinline{assert} and \whileyinline{assume} runtime checks.
Compilation of invalid programs has to fail and yield the expected error message.

To see the performance impact of our lifetime extension on legacy programs, we compare execution times of existing tests with and without the lifetime extension.
The test execution time is the compilation time for the test program and for the valid programs additionally the program's execution time.
To minimize the effect of statistical distribution, we run the tests 25 times each and take the average value for comparison.
We execute the tests on an Intel Core i3 2~GHz dual core processor.

There are 483 existing test cases for valid programs.
The total test execution time for these tests increases from $51.95$ to $57.49$ seconds when using our extension.
This is an increase of $10.66\%$.
Compilation time of the 261 existing invalid programs increases from $9.09$ to $9.81$ seconds, i.e. by $7.92\%$.

\begin{wraptable}{O}{0.65\linewidth}
\footnotesize
\begin{tabular}{ l | c | c | r | r}
test case              & old     & new     & diff $[ms]$ & diff $[\%]$ \\
\hline
FunctionRef\_Valid\_11 & $0.119$ & $0.188$ &  $69$ &  $57.98$ \\
FunctionRef\_Valid\_12 & $0.097$ & $0.212$ & $115$ & $118.56$
\end{tabular}
\caption{Tests with significantly changed execution time}
\label{table:execution-time-significant-change}
\end{wraptable}

We filtered for the tests with an increase of at least  $60ms$ absolutely and at the same time $25\%$ relatively.
this yields only two tests, they are listed in \myref{Table}{table:execution-time-significant-change}.
Both test programs call functions with other functions as parameter, e.g. \whileyinline{f1(&f2)}.
The compiler then has to check whether the type of function \whileyinline{f2} is a subtype of the parameter type of \whileyinline{f1}.
Subtyping on function and method types got more complicated because we have to consider context lifetimes and lifetime parameters.
Even for programs that do not use lifetimes these checks impose some overhead because of added case distinctions.
The general increase in execution time can be explained with an increased size for the abstract syntax tree and type representations.

Our implementation ships with 17 new valid and 14 new invalid programs.
The total execution time for these tests is $2.12$ seconds for the valid and $0.12$ seconds for the invalid programs.
The test cases also cover the method lookup algorithm with omitted lifetime arguments as discussed in \myref{section}{section:substitution-lookup}.
None of these tests takes longer than $0.2$ seconds.
